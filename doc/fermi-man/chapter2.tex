\chapter*{Fermi Input Manual}


%%%%%%%%%%%%%%%%%%%%%%%%%%%%%%%%%%%%%%%%%%%%%%%%%%%%%%%%%%%%%%%%%%%%%%%%%%%%%%%%%%%%

\definecolor{mygreen}{rgb}{0.3, 0.9, 0.1}

%%%%%%%%%%%%%%%%%%%%%%%%%%%%%%%%%%%%%%%%%%%%%%%%%%%%%%%%%%%%%%%%%%%%%%%%%%%%%%%%%%%

\section{Input file}

\subsection{Mesh information}

\begin{Verbatim}[frame=single,commandchars=\\\{\}]
\textcolor{Red}{$mesh}

    \textcolor{OliveGreen}{mesh_file}    cube-multi-materials.msh
    \textcolor{OliveGreen}{dimension}    3
    \textcolor{OliveGreen}{parfile_e}    cube-multi-materials.epart
    \textcolor{OliveGreen}{parfile_n}    cube-multi-materials.npart

\textcolor{Red}{$end_mesh} 
\end{Verbatim}

%%%%%%%%%%%%%%%%%%%%%%%%%%%%%%%%%%%%%%%%%%%%%%%%%%%%%%%%%%%%%%%%%%%%%%%%%%%%%%%%%%%%

\subsection{Energy groups information}

\begin{Verbatim}[frame=single,commandchars=\\\{\}]
\textcolor{Red}{$energy_groups}

    \textcolor{OliveGreen}{numenergy}  1
    \textcolor{OliveGreen}{numprecur}  2
    
\textcolor{Red}{$end_energy_groups}
\end{Verbatim}


%%%%%%%%%%%%%%%%%%%%%%%%%%%%%%%%%%%%%%%%%%%%%%%%%%%%%%%%%%%%%%%%%%%%%%%%%%%%%%%%%%%%

\subsection{Calculation mode}

\begin{Verbatim}[frame=single,commandchars=\\\{\}]
\textcolor{Red}{$calculation_mode}

   \textcolor{OliveGreen}{static} ONLY_ONE 
\textcolor{Gray}{#    static PARAMETRIC {"0",XSA1={0.01,0.01,0.02}}, {"1",XSA1={0.01,0.01,0.02}} }
\textcolor{Gray}{#    transient tf=10.0 dt=0.5 }

\textcolor{Red}{$end_mode}
\end{Verbatim}

%%%%%%%%%%%%%%%%%%%%%%%%%%%%%%%%%%%%%%%%%%%%%%%%%%%%%%%%%%%%%%%%%%%%%%%%%%%%%%%%%%%%

\subsection{Nuclear data information}

\begin{Verbatim}[frame=single,commandchars=\\\{\}]
\textcolor{Red}{$nuclear_data}

   \textcolor{OliveGreen}{xsfile}          xs.fermi
   
\textcolor{Red}{$end_nuclear_data}
\end{Verbatim}

%%%%%%%%%%%%%%%%%%%%%%%%%%%%%%%%%%%%%%%%%%%%%%%%%%%%%%%%%%%%%%%%%%%%%%%%%%%%%%%%%%%%

\subsection{Macroscopic cross sections}


\begin{Verbatim}[frame=single,commandchars=\\\{\}]
\textcolor{Red}{$cross_sections}

     <material name> <0|1> <D> <XSa> <XSs> <nXSf> <eXSf> <chi>
     
\textcolor{Red}{$end_cross_sections}
\end{Verbatim}

\noindent
Where \verb <D> are the diffusion coefficients given as:

\begin{alltt}
<D> = < \(D_{1}\) \(D_{2} \dots D_{g}\) > 
\end{alltt}

\noindent
Where \verb <XSa> is the absortion cross sections given as:

\begin{alltt}
<XSa> = < \(\sigma_{1}^{a}\) \(\nu\sigma_{2}^{a} \dots \nu\sigma_{g}^{a}\) > 
\end{alltt}

\noindent
Where \verb <XSs> are the scattering cross sections given as:

\begin{alltt}
<XSs> = < \(\sigma_{2\rightarrow1}\) \(\sigma_{3\rightarrow1}  \dots \sigma_{{g\text{-}1}\rightarrow{g}}\) > 
\end{alltt}

\noindent
Where \verb <nXSf> are the number of neutrons emitted per fission time the fission cross sections given as:

\begin{alltt}
<nXSf> = <  \(\nu\sigma_{1}^{f}\) \(\nu\sigma_{2}^{f} \dots \nu\sigma_{g}^{f}\) > 
\end{alltt}

\noindent
Where \verb <eXSf> are the energy per fission times the fission cross sections given as:

\begin{alltt}
<eXSf> = <  \(e\sigma_{1}^{f}\) \(e\sigma_{2}^{f} \dots e\sigma_{g}^{f}\) > 
\end{alltt}

\noindent
Where \verb <chi> is the fission spectrum given as:

\begin{alltt}
<chi> = < \( \chi_{1} \) \(\chi_{2}\dots\chi_{g}\) > 
\end{alltt}

%%%%%%%%%%%%%%%%%%%%%%%%%%%%%%%%%%%%%%%%%%%%%%%%%%%%%%%%%%%%%%%%%%%%%%%%%%%%%%%%%%%%

\subsection{Fission precursors' constants}

\begin{Verbatim}[frame=single,commandchars=\\\{\}]
\textcolor{Red}{$precursor_constants}

   <beta> <lambda> <chi>
   
\textcolor{Red}{$end_cross_sections}
\end{Verbatim}

\noindent
Where \verb <beta>  are the fission precursors yields given as:

\begin{alltt}
<beta> = < \(\beta_{1}\) \(\beta_{2} \dots \beta_{G}\) > 
\end{alltt}

\noindent
Where \verb <lambda>  are the precursors' decay constants given as:

\begin{alltt}
<beta> = < \(\lambda_{1}\) \(\lambda_{2} \dots \lambda_{G}\) > 
\end{alltt}

\noindent
Where \verb <chi>  are the precursors' neutron energy spectrum given as:

\begin{alltt}
<chi> = < \(\chi_{11} \chi_{21} \dots \chi_{G1} \chi_{12} \dots \chi_{Gg}\) > 
\end{alltt}

%%%%%%%%%%%%%%%%%%%%%%%%%%%%%%%%%%%%%%%%%%%%%%%%%%%%%%%%%%%%%%%%%%%%%%%%%%%%%%%%%%%%

\subsection{Outputs}


Is it possible to print variables on ASCII files in order to obtain results from the code
directly. There are other forms of obtaining results form the code that is coupling it with
other codes and to perform coupling calculations. We recommend this last one but the output
mode is here to provide the user with quick results without spending to much time on code 
coupling.

\begin{Verbatim}[frame=single,commandchars=\\\{\}]
   \textcolor{Red}{$Output}
       \textcolor{OliveGreen}{kind <kind>}
       \textcolor{OliveGreen}{<arguments>}
   \textcolor{Red}{$EndOutput}
       \textcolor{OliveGreen}{file <file>}
       \textcolor{OliveGreen}{nphy <N>}
       \textcolor{OliveGreen}{<physical\_1>  <physical\_2> \dots <physical\_N>}
\end{Verbatim}

Where \textalltt{\textcolor{OliveGreen}{<kind>}} is an integer number code that represents 
the kind of output that is going to be print. The values of 
\textalltt{\textcolor{OliveGreen}{<arguments>}} depends on the last value.Possible values 
are:

\begin{itemize}
   \item 1: do nothing
   \item 2: prints powers on physical entities on ASCII file in each time step.
\begin{Verbatim}[frame=single,commandchars=\\\{\}]
       \textcolor{OliveGreen}{file <file>}
       \textcolor{OliveGreen}{nphy <N>}
       \textcolor{OliveGreen}{<physical\_1>  <physical\_2> \dots <physical\_N>}
\end{Verbatim}
\end{itemize}


%%%%%%%%%%%%%%%%%%%%%%%%%%%%%%%%%%%%%%%%%%%%%%%%%%%%%%%%%%%%%%%%%%%%%%%%%%%%%%%%%%%%

\subsection{Communication}

It is possible to establish communication with other codes in order to perform couple 
calculations. This is one of the most important properties of \fermi code. The communication
is performs thanks to the \verb PLEPP  library.

\begin{Verbatim}[frame=single,commandchars=\\\{\}]
   \textcolor{Red}{$Communication}
       \textcolor{OliveGreen}{kind <kind>}
       \textcolor{OliveGreen}{<arguments>}
   \textcolor{Red}{$EndCommunication}
\end{Verbatim}

Where \textalltt{\textcolor{OliveGreen}{<kind>}} is an integer number code that represents 
the kind of communication that is going to be done. The values of 
\textalltt{\textcolor{OliveGreen}{<arguments>}} depends on the 
\textalltt{\textcolor{OliveGreen}{<kind>}} value.Possible values 
are:

\begin{itemize}
   \item 1: communication with one code where the cross sections are passed at each time 
   step and \fermi returns the power corresponding to that evolution.
\begin{Verbatim}[frame=single,commandchars=\\\{\}]
       \textcolor{OliveGreen}{friend <friend>}
       \textcolor{OliveGreen}{nphy <N>}
       \textcolor{OliveGreen}{<physical\_1>  <physical\_2> \dots <physical\_N>}
\end{Verbatim}
\end{itemize}

